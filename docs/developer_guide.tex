\documentclass{article}
\usepackage{listings}
\usepackage{color}
\usepackage{url}

\begin{document}
\title{Developer's Guide to \emph{Flow}}
\author{Cliburn Chan}
\date{\today}
\maketitle

\section*{Introduction}

\emph{Flow} is a software package written in Python for exploratory
data analysis, clustering and classification of flow cytometric
data. While the base system provides the data pre-processing and data
management tools, most of the interesting functionality of \emph{Flow}
is provided by plugins. Currently, \emph{Flow} accepts four classes of
plugins – IO (input/output), Projections, Statistics and
Visualization. This document describes how to develop plugins for each
of these classes, as well as how to develop plugins in R and compiled
languages like C/C++/Fortran. Familiarity with Python is necessary to
follow from this guide.

\section*{Plugins}

Plugins can be implemented either as a single file (for simple
projects) or as a directory of files (for more complex projects). Each
directory must contain a file named \texttt{Main.py} that is the entry
point for the plugin. Each standalone plugin or \texttt{Main.py}
program has a class that subclasses one of the provided IO,
Projection, Statistics or Visualization classes and define certain
mandatory attributes depending on the class of plugin it belongs
to. This will be clear by following the specific examples provided.

In general, plugins will also have to interface with the Model class
where the data is stored in order to do anything useful. The Model
class provides several methods for data retrieval in its API (see
Doxygen generated API).

\subsection*{Developing IO plugins}

For our first plugin, we will add the capacity to read data from a
comma-separated file (CSV). The basics of reading data from any source
is essentially the same:

\begin{enumerate}
\item Extract a list of field names and convert the numerical data
  into a numpy array
\item Load the data into the current Group (creating it first if necessary)
\end{enumerate}

It is trivial to implement the first step for CSV files, and the Model
class provides a utility method for the second step. Start by creating
a new file ReadCSV.py, which will become the new plugin. We start by
importing some Python modules

\lstset{language=python}
\begin{lstlisting}[]{}
from io import Io
import numpy
import os
\end{lstlisting}

All IO plugins need to subclass the base class \texttt{Io} to let the
system know that we are defining a new IO class. The module
\texttt{numpy} provides efficient numerical support for Python, and we
will store our data as numpy array. The standard module \texttt{os} is
used to manipulate filenames in a system-independent fashion.

Now we define the actual plugin class, which is specified as a
subclass of IO and called \texttt{ReadCSV}. This has several class
attributes and a single method, also called \texttt{ReadCSV} to be
described following the code listing.

\begin{lstlisting}
class ReadCSV(Io):
  """An IO plugin to read CSV files in which 
  the first line consists of field names and 
  subsequent lines consist of channel data."""

  type = 'Read'
  newMethods = ('ReadCSV', 'Load CSV data file')
  supported = 'CSV files (*.csv)|*.csv|All files (*.*)|*.*'
  
  def ReadCSV(self, filename):
    """Reads a CSV file and populates data structures."""
    
    # read and parse comma-separated header and data
    text = open(filename, 'rU').readlines()
    headers = text[0].strip('\n').split(',')
    arr = numpy.array([map(float, line.strip().split(',')
                       for line in text[1:])])

    # create a new group using the filename base as label
    basename = os.path.basename(filename)
    base, ext = os.path.splitext(basename)
    self.model.NewGroup(base)

    # put the data in the current group
    self.model.LoadData(headers, arr, 'Original data')

\end{lstlisting}

The \texttt{type} attribute specifies whether the plugin will read
(import) or write (export) data, and can take the values \texttt{Read}
and \texttt{Write}. In general, the \texttt{type} attribute specifies
how the front-end will package the functionality (for example, menu
placement or addition to contextual menus).

The code to read and parse CSV data should be clear to any Python
programmer, and results in a string of field names stored in
\texttt[headers] and the channel data stored as a numpy array in
\texttt[arr].

Finally, we create a new Group (a Group is a non-terminal node in
HDF5) to store the new data, give it the same name as the data
filename (stripping off the file extension if present), and load it
into the model using the provided hook \texttt{LoadData}.

If you've followed and written the code, you've just written your
first plugin for \emph{Flow}. To test the plugin, write a test CSV
file like this

\begin{verbatim}
foo, bar, one, two, three
1,2,3,4,5
2,3,4,5,6
3,4,5,6,7
4,5,6,7,8
1.5,2.5,3.5,4.5,6.5
\end{verbatim}

and save it as \texttt{testfile.csv}. Now move \texttt{ReadCSV.py} to
the \emph{plugins/io} sub-directory, start \emph{Flow} and there will
be a new menu item \texttt{File|Load CSV data file} that allows you to
browse for and open \texttt{textfile.csv}. After loading the file, the
Control Frame will show a new group under the root labeled
\texttt{testfile}. Expand the group by clicking on the horizontal
triangle to show the array \texttt{data}. Selecting \texttt{data} will
show its associated metadata (in this case, only the field names are
useful), and you can plot the data using one of the \texttt{Graphics}
menu options etc.

\subsection*{Developing Statistics plugins}

The process of developing statistics plugins is very similar, and only
a trivial example will be shown here, as realistically, most
statistics plugins will either be written in a compiled language
(C/C++/Fortran) or interface with the R statistical libraries. Such
foreign language plugins will be described later.

We will develop a statistics plugin to calculate the mean value of
each column -- while trivial, this demonstrates how to retrieve data
from the Model, process it and append new results to the Model.

We begin by making the necessary imports

\begin{lstlisting}[]{}
from plugin import Statistics
import numpy
\end{lstlisting}

As before, our plugin class needs to subclass \texttt{Statistics},
then it is a simple matter of calling the Model API to retrieve data,
find its mean across columns and append the calculated statistic.

\begin{lstlisting}
class Mean(Statistics):
  """Calculate channel averages and 
  append statistic to group."""
  name = 'Average'

  def Main(self, model):
    """Retrieve column data from Model and 
    attach calculated column means."""

    # make a copy of the currently selected group's data
    data = model.GetCurrentData()[:]

    # calculate the mean per channel
    avg = numpy.mean(data, axis=0)

    # make a new array to store the calculated means
    model.NewArray('average', avg)

\end{lstlisting}

Copy this file to the \emph{plugins/statistics} sub-directory, fire up
\emph{Flow}, and apply the new menu item \texttt{Statistics|Average}
to the \texttt{testfile} data. This generages a new sub-group in the
Control Frame with the label \texttt{average}. Right clicking and
choosing \texttt{Edit} will show the calculated channel averages in a
table.

\subsection*{Developing R plugins}

Thanks to the \texttt{RPy} library, it is generally very simple to
write an R plugin for \emph{Flow}. We will illustrate how to find the
independent components of the data using the R library
\texttt{fastICA}. We assume that the user has a working R installation
and has installed the \texttt{fastICA} library (see the R
documentation for detaails at \url{http://www.r-project.org}).

Begin, as before, with the imports, and load the \texttt{fastICA}
library

\begin{lstlisting}[]{}
from plugin import Projections
from rpy import r
import wx
import numpy
r.library("fastICA")
\end{lstlisting}

Now create a class that will do the necessary calculation and
communication

\begin{lstlisting}
class Ica(Projections):
  """Uses the fastICA library to find 
  independent components."""

  def Main(self, model):
    k = wx.GetNumberFromUser("ICA Dialog",
                             "Enter number of components",
                             "k", 1)
    data = numpy.array(model.GetCurrentData()[:])
    ica_data = r.fastICA(data, k)
    fields = ['Comp%d' % c for c in range(1, k+1)]
    model.updateHDF('ICA', ica_data['S'], fields=fields)
\end{lstlisting}

As can be seen, the class is almost trivial. We ask the user for the
number of components $k$ desired using a standard wxPython dialog,
then pass a copy of the current data and $k$ to R, and update the
Model with the result and new appropriate field names. See the
\texttt{fastICA} documentation for more details
(\url{http://cran.r-project.org/web/packages/fastICA}) of its
functionality.

\subsection*{Developing compiled language plugins}

Sometimes, Python and R are just too slow and we need the speed of a
compiled language like C, C++ or Fortran, but still want to use
\emph{Flow} to provide a frontend to these routines. Before plunging
in, do check if Python optimization tricks will be enough -- see guide
at \url{http://www.scipy.org/PerformancePython} for examples. 

We will not actually develop an example of a compiled language
extension here, but merely suggests possible routes and resources. To
allow \emph{Flow} to interface with C or C++, a typical strategy is to
compile a Python extension module, following the instructions at
\url{http://www.python.org/doc/ext/intro.html}. Alternative and
simpler methods include using Swig (http://www.swig.org), and for C++
only, using the Boost.Python library
(\url{http://www.boost.org/libs/python }). For Fortran, we recommend
using F2PY (\url{http://cens.ioc.ee/projects/f2py2e}). C++ examples
can be found in the \emph{c++} sub-directory.

Once an extension module is compiled into a shared library, it can be
imported into Python like any other module and the subsequent plugin
development is as already described for the various plugin classes.

\subsection*{Developing Visualization plugins}

Writing a visualization plugin is rather more involved than the
previous plugins, and requires knowledge of the specific GUI toolkits
to be used, and will also not be described here. Interested developers
will have to look at the provided source code examples in the
\emph{plugins/visualization} for now.

\end{document}

%%% Local Variables: 
%%% mode: latex
%%% TeX-master: t
%%% End: 
